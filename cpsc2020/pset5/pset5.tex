\documentclass[12pt, letterpaper]{article}
\title{pset 5}
\author{Jaden Lee\thanks{made in Overleaf}}
\date{2/19/26}
\usepackage{hyperref}
\usepackage{graphicx}
\usepackage{float}
\usepackage{outlines}

\usepackage{amssymb}
\usepackage{tabularx}
\newcommand{\Z}{\mathbb{Z}}
\newcommand{\N}{\mathbb{N}}
\renewcommand{\arraystretch}{1.25}
\begin{document}
\maketitle
\hyperlink{https://yale.instructure.com/courses/113840/assignments/558953}{https://yale.instructure.com/courses/113840/assignments/558953}
% \neg for neg
% \wedge for and 
% \vee for or
% \rightarrow
% \equiv

% https://www.tablesgenerator.com/# <-- latex table generator

\section{Problem 1}
Our goal is to find the smallest k so that can we can make k' cents postage for any integer $k' \geq k$ with 4-, 9- and 15-cent stamps. (
k and k' are restricted to be integers throughout the prompts given below.)

\subsection{solution}

\begin{outline}[enumerate]
    \1 Prove that if you can make k cents postage with some combination of 4-, 9-, and 15-cent stamps that includes at least 2 4-cent stamps, then you can make k+1 cents postage
        \2 suppose there $\exists k \in \Z [4a + 9b + 15c = k]$ and that $ a \geq 2$
        
        let a' = a-2 and b' = b+1 and c' = c and k' = 4a' + 9b' + 15c'; also that a',b',c',k' $\exists \in \N$
        so $$k' = 4(a-2) + 9 (b+1) + 15c$$
        $$= 4a - 8 + 9b + 9 + 15c$$
        $$= k - 8 + 9 = k + 1$$
        conclusion: you can make k+1 cents postage given that you have at least 2 4-cent postages

    \1 Prove that if you can make k  cents postage with some combination of 4-, 9-, and 15-cent stamps that includes at least 2 9-cent stamps, then you can make k+1 cents postage
        \2 suppose there $\exists k \in \Z [4a + 9b + 15c = k]$ and that $ b \geq 2$
            
        let a' = a+1 and b' = b-2 and c' = c+1 and k' = 4a' + 9b' + 15c'; also that a',b',c',k' $\exists \in \N$
        so $$k' = 4(a+1) + 9 (b-2) + 15 (c+1)$$
        $$= 4a + 4 + 9b - 18 + c + 15$$
        $$= k - 18 + 19 = k + 1$$
        conclusion: you can make k+1 cents postage given that you have at least 2 9-cent postages
    
    \1 Prove that if you can make k  cents postage with some combination of 4-, 9-, and 15-cent stamps that includes at least 1 15-cent stamps, then you can make k+1 cents postage

        \2 suppose there $\exists k \in \Z [4a + 9b + 15c = k]$ and that $ c \geq 1$
                
        let a' = a+4 and b' = b and c' = c-1 and k' = 4a' + 9b' + 15c'; also that a',b',c',k' $\exists \in \N$
        so $$k' = 4(a+4) + 9 (b) + 15 (c-1)$$
        $$= 4a + 16 + 9b + c - 15$$
        $$= k - 15 + 16 = k + 1$$
        conclusion: you can make k+1 cents postage given that you have at least 1 15-cent postages
    
    \1 Prove that if you have k  cents postage for some $k \geq 14$ then you must have either 1) at least 2 4-cent stamps; 2) at least 2 9-cent stamps; or 3) at least 1 15-cent stamp.

        \2 $\therefore $ where $\exists a,b,c \in \N [k = 4a+9b+15c \geq 14]$

        base case: a,b,c = 0; then k < 14 so cannot be true
            \3 lemma: if $c \geq 1 \rightarrow k \geq 14$
                in this case, $k \geq 15 \cdot c$
                $$= 15 \geq 14$$

            \3 cases where 2 of a,b,c are 0 but one is not.
            
                \4 a $\neq$ 0; note that if $a < 4, k < 14$ by algebra (in this case, k = 4a); by contrapositive (modus ponens), $a \geq 4 > 2$ in this case
                
                \4 b $\neq$ 0; note that if $b < 2, k < 14$ by algebra (in this case, k = 9b); by contrapositive (modus ponens), $b \geq 2$ in this case

                \4 c $\neq$ 0; see lemma

            \3 cases where 1 of a,b,c are 0 but two are not.

                \4 a,b $\neq$ 0;$k = 4a + 9b$ if $a \leq 1\wedge b$ $\leq 1$, then k $<$ 14. By contrapositive (modus ponens), one of a,b must be $\geq 2$ (individually; in other words, $a \geq 2 \vee b \geq 2$)

                \4 b,c $\neq$ 0; see lemma

                \4 c,a $\neq$ 0; see lemma

            \3 case where none are 0

                \4 see lemma
    
    \1 Find the smallest k so that you can make k' cents postage for any $k' \geq k$ with 4-, 9-, and 15-cent stamps.

        \2 suppose that $\exists a,b,c \in \N [k = 4a+9b+15c]$
        
        this is a case of $p \vee q \vee r$ where 
            \3 $p \equiv a \geq 2$
            \3 $q \equiv b \geq 2$
            \3 $r \equiv c \geq 1$
        (if any of the above conditions are met, k+1 can be performed.)

        furthermore, k+1 must also contain these conditions ($p \vee q \vee r$ must also apply to k+1)

        method: find smallest of the base cases

        \2 
        strong induction: for all natural numbers p(i) > 15, if p(n) is true for n = 18, … , k - 1
        then p(k) is true
            \3 base case: is p(n) true for 15? yes, because $15 = 15 \cdot 1$
            \3 base case: is p(n) true for 16? yes, because $16 = 4 \cdot 4$
            \3 base case: is p(n) true for 17? yes, because $17 = 4 \cdot 2 + 9$
            \3 base case: is p(n) true for 18? yes, because $18 = 9 \cdot 3$

    \1 For your chosen k , prove that you can make k' cents postage for any $k' \geq k$ with some combination of 4-, 9-, and 15-cent stamps.

    (continued from proof above)
        \2 strong induction: for all natural numbers p(i) > 15, if p(n) is true for n = 15, … , k - 1
        then p(k) is true
        
            \3 suppose p(i) is true for all i where $18 \leq i \leq k-1$
            p(k) = 4a + 9b + 15c

            from 15... 18, you can add 4 to each and generate the next 4 digits... in other words:
           $$ k+1 = k-3 + 4$$
           $$ = k-3 + 4 \cdot a$$ where a =
           
\end{outline}



\section{problem 2}

\begin{figure}[H]
    \centering
    \includegraphics[width=1\linewidth]{cpsc2020//pset5/p2.png}
    \caption{problem 2}
    \label{fig:placeholder}
\end{figure}

\subsection{solution}

\textbf{base case: }
$ n = 0$
$$\sum_{i=1}^{n} 2i+3 = n(n+4)$$
$$= \sum_{i=1}^{0} 2i+3 = n(n+4)$$
$$0 = 0$$ 
(empty set rule used above)

\textbf{inductive step: }suppose $n \geq 0$, and the rule applies to n
$$\sum_{i=1}^{n+1} 2i+3 = \sum_{i=1}^{n} 2i+3 + (2(n+1) + 3)$$
$$= n(n+4) + (2(n+1) + 3)$$
$$= n^2 + 4n + (2(n+1) + 3)$$
$$= n^2 + 6n + 5$$
$$= (n+1)(n+5)$$
$$= (n+1)((n+1) +4)$$



\section{problem 3}
\begin{figure}[H]
    \centering
    \includegraphics[width=1\linewidth]{cpsc2020//pset5/problem 3.png}
    \caption{problem 3}
    \label{fig:placeholder}
\end{figure}

\subsection{solution}
\textbf{base case: } n = 1
$$a_1 = 8$$ $$8 \mid 8$$

\textbf{induction step: } suppose for all $n \geq 1$, $odd(n) \rightarrow 8 \mid a_n$

$$a_{n+2} = [a_{n}  + 4(n+2) - 2] + 4((n+1) + 1) - 2$$
$$ = [a_{n}  + 4n + 2] + 4(n+2) -2$$
$$ = a_{n}  + 4n + 2 + 4n + 6 $$
$$ = a_{n}  + 8n + 8 $$

since $8 \mid a_n \wedge 8 \mid 8n \wedge 8 \mid 8$
and since $n + 2$ is odd
$8 \mid a_{n +2}$ given n is an odd natural number

\section{problem 4}
\begin{figure}[H]
    \centering
    \includegraphics[width=1\linewidth]{cpsc2020//pset5/problem 4.png}
    \caption{problem 4}
    \label{fig:placeholder}
\end{figure}


\subsection{solution}

\textbf{base cases}: 

$a_0 = 4$ (mod 6)

$a_1 = 16$ = 4 (mod 6)

\textbf{induction step (stronk)}:

for some index i where $\exists i \in  \N [0 \leq i \leq n-1]$ and that p(i) is true (that $a_i = 4$ (mod 6)

$ a_n = 2 \cdot a_{n-1} + a_{n-2} - 2$

$a_n$ (mod 6) $ = 2 \cdot $ 4 (mod 6) + 4 (mod 6) - 2 (mod 6)

 = 8 (mod 6) + 4 (mod 6) - 2

 = 2 (mod 6) - 2 (mod 6) + 4 (mod 6)
 
 = 4 (mod 6)

 $a_n$ = 4 (mod 6) for all integers $n \geq 0$


\end{document}