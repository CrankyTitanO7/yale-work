\documentclass[12pt, letterpaper]{article}
\title{pset 4}
\author{Jaden Lee\thanks{made in Overleaf}}
\date{2/12/26}
\usepackage{hyperref}
\usepackage{amssymb}
\usepackage{float}
\usepackage{tabularx}
\newcommand{\Z}{\mathbb{Z}}
\renewcommand{\arraystretch}{1.25}
\begin{document}
\maketitle
\hyperlink{https://yale.instructure.com/courses/113840/assignments/558952}{https://yale.instructure.com/courses/113840/assignments/558952}
% \neg for neg
% \wedge for and 
% \vee for or
% \rightarrow
% \equiv
% https://www.tablesgenerator.com/# <-- latex table generator

\section{Problem 1}
Problem 1 (6 points): Prove that for any three digit positive integer, if the sum of the digits is a multiple of 3, then the number is a multiple of 3. (Hint: if the hundreds digit of $n$ is $a$ and the tens digit is $b$ and the ones digit is $c$, what is $n$ equal to in terms of $a, b,$ and $c$?)

\subsection{solution}
\begin{table}[H]
\centering
\begin{tabularx}{\textwidth}{p{0.8cm} >{\raggedright\arraybackslash}X p{4cm}}
1 & Suppose $c$ is a three-digit positive integer with digits $x, y, z$ & supposition \\
2 & $c = 100 \cdot x + 10 \cdot y + z$ & representation of 3-digit number \\
3 & Suppose $3 \mid (x + y + z)$ & supposition (given hypothesis) \\
4 & $100 \equiv 1 \pmod{3}$ & computation: $100 = 3(33) + 1$ \\
5 & $10 \equiv 1 \pmod{3}$ & computation: $10 = 3(3) + 1$ \\
6 & $c = 100x + 10y + z \equiv 1 \cdot x + 1 \cdot y + z \pmod{3}$ & modular arithmetic (lines 4, 5) \\
7 & $c \equiv x + y + z \pmod{3}$ & algebra (line 6) \\
8 & Since $3 \mid (x+y+z)$, we have $x + y + z \equiv 0 \pmod{3}$ & definition of divisibility \\
9 & $c \equiv 0 \pmod{3}$ & substitution (lines 7, 8) \\
10 & $\therefore 3 \mid c$ & definition of congruence \\
\end{tabularx}
\end{table}

\section{problem 2}
(6 points): Use the modular arithmetic corollary of the Quotient/Remainder Theorem to prove that for all integers $n, n^8 \equiv 0 \pmod{5}$ or $n^8 \equiv 1 \pmod{5}$

\subsection{solution}
By the Modular Arithmetic Corollary of the Quotient/Remainder Theorem, every integer $n$ satisfies exactly one of: $n \equiv 0, 1, 2, 3, 4 \pmod{5}$. We examine each case:

\begin{table}[H]
\centering
\begin{tabularx}{\textwidth}{p{0.8cm} >{\raggedright\arraybackslash}X p{4cm}}
1 & \textbf{Case 1:} $n \equiv 0 \pmod{5}$ & supposition \\
2 & $n^8 \equiv 0^8 \equiv 0 \pmod{5}$ & exponentiation \\
\end{tabularx}
\end{table}

\begin{table}[H]
\centering
\begin{tabularx}{\textwidth}{p{0.8cm} >{\raggedright\arraybackslash}X p{4cm}}
3 & \textbf{Case 2:} $n \equiv 1 \pmod{5}$ & supposition \\
4 & $n^8 \equiv 1^8 \equiv 1 \pmod{5}$ & exponentiation \\
\end{tabularx}
\end{table}

\begin{table}[H]
\centering
\begin{tabularx}{\textwidth}{p{0.8cm} >{\raggedright\arraybackslash}X p{4cm}}
5 & \textbf{Case 3:} $n \equiv 2 \pmod{5}$ & supposition \\
6 & $n^2 \equiv 2^2 \equiv 4 \equiv -1 \pmod{5}$ & computation \\
7 & $n^4 \equiv (n^2)^2 \equiv (-1)^2 \equiv 1 \pmod{5}$ & exponentiation (line 6) \\
8 & $n^8 \equiv (n^4)^2 \equiv 1^2 \equiv 1 \pmod{5}$ & exponentiation (line 7) \\
\end{tabularx}
\end{table}

\begin{table}[H]
\centering
\begin{tabularx}{\textwidth}{p{0.8cm} >{\raggedright\arraybackslash}X p{4cm}}
9 & \textbf{Case 4:} $n \equiv 3 \pmod{5}$ & supposition \\
10 & $n^2 \equiv 3^2 \equiv 9 \equiv 4 \equiv -1 \pmod{5}$ & computation \\
11 & $n^4 \equiv (-1)^2 \equiv 1 \pmod{5}$ & exponentiation (line 10) \\
12 & $n^8 \equiv 1^2 \equiv 1 \pmod{5}$ & exponentiation (line 11) \\
\end{tabularx}
\end{table}

\begin{table}[H]
\centering
\begin{tabularx}{\textwidth}{p{0.8cm} >{\raggedright\arraybackslash}X p{4cm}}
13 & \textbf{Case 5:} $n \equiv 4 \pmod{5}$ & supposition \\
14 & $n \equiv -1 \pmod{5}$ & since $4 \equiv -1 \pmod{5}$ \\
15 & $n^8 \equiv (-1)^8 \equiv 1 \pmod{5}$ & exponentiation \\
\end{tabularx}
\end{table}

\begin{table}[H]
\centering
\begin{tabularx}{\textwidth}{p{0.8cm} >{\raggedright\arraybackslash}X p{4cm}}
16 & $\therefore \forall n \in \Z: n^8 \equiv 0 \pmod{5} \vee n^8 \equiv 1 \pmod{5}$ & exhaustive cases (lines 1-15) \\
\end{tabularx}
\end{table}

\section{problem 3}
(8 points): Prove that, for any integers $n, a, b, c, d$ with $n \geq 2$ and $a \equiv b \pmod{n}$ and $c \equiv d \pmod{n}$, then $ac \equiv bd \pmod{n}$

\subsection{solution}
\begin{table}[H]
\centering
\begin{tabularx}{\textwidth}{p{0.8cm} >{\raggedright\arraybackslash}X p{4cm}}
1 & Suppose $n \geq 2$ and $a, b, c, d \in \Z$ & supposition \\
2 & Suppose $a \equiv b \pmod{n}$ and $c \equiv d \pmod{n}$ & supposition (given) \\
3 & $n \mid (a-b)$ and $n \mid (c-d)$ & definition of congruence \\
4 & $\exists k, m \in \Z$ s.t. $a - b = nk$ and $c - d = nm$ & definition of divisibility \\
5 & $a = b + nk$ and $c = d + nm$ & algebra (line 4) \\
6 & $ac = (b + nk)(d + nm)$ & substitution (line 5) \\
7 & $ac = bd + bnm + dnk + n^2km$ & algebra (expansion) \\
8 & $ac = bd + n(bm + dk + nkm)$ & algebra (factoring) \\
9 & $ac - bd = n(bm + dk + nkm)$ & algebra (line 8) \\
10 & Since $bm + dk + nkm \in \Z$, we have $n \mid (ac - bd)$ & definition of divisibility \\
11 & $\therefore ac \equiv bd \pmod{n}$ & definition of congruence \\
\end{tabularx}
\end{table}

\section{problem 4}
(8 points): Complete the proof that the Euclidean algorithm is correct by showing that for all integers $a, b, r$ if $b \neq 0$ and $a = b \cdot q + r$ for some integer q, then $\gcd(b,r) \leq \gcd(a,b).$

\subsection{solution}
\begin{table}[H]
\centering
\begin{tabularx}{\textwidth}{p{0.8cm} >{\raggedright\arraybackslash}X p{4cm}}
1 & Suppose $a, b, r, q \in \Z$ with $b \neq 0$ & supposition \\
2 & Suppose $a = bq + r$ & supposition (given) \\
3 & Let $d = \gcd(b, r)$ & definition \\
4 & $d \mid b$ and $d \mid r$ & definition of gcd (line 3) \\
5 & Since $d \mid b$, we have $d \mid (bq)$ & divisibility property \\
6 & Since $d \mid r$ and $d \mid (bq)$, we have $d \mid (bq + r)$ & divisibility of sums \\
7 & $d \mid a$ & substitution (lines 2, 6) \\
8 & Therefore $d$ is a common divisor of $a$ and $b$ & lines 4, 7 \\
9 & Since $\gcd(a,b)$ is the \emph{greatest} common divisor, $d \leq \gcd(a,b)$ & definition of gcd \\
10 & $\therefore \gcd(b,r) \leq \gcd(a,b)$ & substitution (lines 3, 9) \\
\end{tabularx}
\end{table}

\end{document}