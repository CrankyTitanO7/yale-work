\documentclass[12pt, letterpaper]{article}
\title{pset 3}
\author{Jaden Lee\thanks{made in Overleaf}}
\date{2/5/26}
\usepackage{hyperref}
\usepackage{amssymb}
\usepackage{float}
\usepackage{tabularx}

\newcommand{\Z}{\mathbb{Z}}
\renewcommand{\arraystretch}{1.25}

\begin{document}
\maketitle
\hyperlink{https://yale.instructure.com/courses/113840/assignments/558951}{https://yale.instructure.com/courses/113840/assignments/558951}
% \neg for neg
% \wedge for and 
% \vee for or
% \rightarrow
% \equiv

% https://www.tablesgenerator.com/# <-- latex table generator

\section{Problem 1}

(25 points): Write each of the following in predicate logic, using only quantifiers (over the set of integers $\Z$, logical operators, arithmetic operators ( $+, -, \cdot$), and relational operators ($<, >, =$, or combinations thereof). Write the negation using the same restrictions, and the additional restriction that there are no negations in front of the quantifiers. Determine which of the negation and the original statement is true (you need not give a proof).

\begin{itemize}
    \item If two integers sum to something greater than 30, then at least one of them is greater than 15.
    \item Any integer that is a multiple of both 4 and 6 is also a multiple of 24.
    \item Between any two odd integers there is some even integer.
    \item Between any two distinct odd integers there is some even integer.
    \item There is a largest even integer.
\end{itemize}

\subsection{solution} \footnote{note that solutions use odd() and even(). the exact proof predicate definitions are listed at the end, and are only omitted here to manage clutter}

\begin{itemize} 
    \item \textbf{Original is True}
        $$\forall i, k \in \Z [(i + k > 30) \rightarrow (i > 15 \vee k > 15)]$$
        $$\exists i, k \in \Z [(i + k > 30) \wedge (i \leq 15 \wedge k \leq 15)]$$

    \item \textbf{Negation is True}
        $$\forall i \exists k, l,m\in \Z [(4k=i \wedge 6l=i) \rightarrow 24m= i]$$
        $$\exists i, k, l\in \Z [4l=i \wedge 6l= i \wedge 24m \neq i]$$

    \item \textbf{Negation is True} (False if $x=y$)
        $$\forall x, y \in \Z [(Odd(x) \wedge Odd(y)) \rightarrow \exists m \in \Z (Even(m) \wedge (x < m < y \vee y < m < x))]$$
        $$\exists x, y \in \Z [Odd(x) \wedge Odd(y) \wedge \forall m \in \Z (Even(m) \rightarrow (m \leq x \vee m \geq y) \wedge (m \leq y \vee m \geq x))]$$

    \item \textbf{Original is True}
        $$\forall x, y \in \Z [(Odd(x) \wedge Odd(y) \wedge x \neq y) \rightarrow \exists m \in \Z (Even(m) \wedge (x < m < y \vee y < m < x))]$$
        $$\exists x, y \in \Z [Odd(x) \wedge Odd(y) \wedge x \neq y \wedge \forall m \in \Z (Even(m) \rightarrow (m \leq x \vee m \geq y) \wedge (m \leq y \vee m \geq x))]$$

    \item \textbf{Negation is True}
        $$\exists x \in \Z [Even(x) \wedge \forall y \in \Z (Even(y) \rightarrow x \geq y)]$$
        $$\forall x \in \Z [Even(x) \rightarrow \exists y \in \Z (Even(y) \wedge y > x)]$$.
\end{itemize}



% \cdot for multiply
% \mid for divisible by


\section {problem 2}
(25 points): Write proofs of each of the following, giving a justification of each step that includes the rule of inference, logical equivalence, axiom, or theorem used (permitted theorems include rules of arithmetic or algebra, previous results, and the following: every integer is either even or odd but not both).  (Hint: for some of these, it may be easier to prove the contrapositive.)
\begin{itemize}
    \item For every odd integer, the next integer is even.
    \item If three positive integers sum to 100, then at least one of them is greater than 33.
    \item If the product of two integers is odd, then both integers are odd.
    \item The product of two consecutive integers is even.
    \item For any integer $x$, if $6 \mid x^2$, then $4 \mid x^3-x^2$
\end{itemize}

\subsection{solution}


\subsubsection{For every odd integer, the next integer is even.}

\begin{table}[H]
\centering
\begin{tabularx}{\textwidth}{p{0.8cm} >{\raggedright\arraybackslash}X p{4cm}}
1 &suppose integer j such that $\exists j,x \in \Z [j = 2x + 1]$ &supposition \\
2 &$j = 2x + 1$ &restating 1 in formula format \\
3 &suppose j + 1 is also odd &supposition \\
4 &$2\nmid j+1$ &restating 3 in formula format \\
5 &$2 \nmid 2x + 2$ &algebra \\
6 &$2 \nmid 2(x +1)$ is a false statement&algebra \\
7 &$(x +1) \in \Z$ &closure \\
8 &j+1 is not odd &definition of even \\
9 &j+1 is even &definition of even x2 \\
17 &$\therefore$ For every odd integer (j), the next integer is even (j + 2). &conclusions 2 \\
\end{tabularx}
\end{table}

\subsubsection{If three positive integers sum to 100, then at least one of them is greater than 33.}

\begin{table}[H]
\centering
\begin{tabularx}{\textwidth}{p{0.8cm} >{\raggedright\arraybackslash}X p{4cm}}
1 &suppose $\exists x, y, z \in \Z [x + y + z =100]$ &supposition \\
2 &suppose $x < 33$ &supposition \\
3 &then $y + z > 67$ &algebra (based on 100-x is between 67 and 100) \\
4 &suppose $y < 33$ &supposition \\
5 &$z < 33 \rightarrow y + z \leq 67$; &algebra (based on 67-y is between 34 and 67) \\
6 &then $z > 33$; in other words, z must be $>33$ otherwise $z + y < 67$ &conclusions (for supposition in 4) \\
7 &then $z > 33$; in other words, z must be $>33$ otherwise $x + y + z < 100$ &conclusions (for supposition in 2) \\
\end{tabularx}
\end{table}

\subsubsection{If the product of two integers is odd, then both integers are odd.}

\begin{table}[H]
\centering
\begin{tabularx}{\textwidth}{p{0.8cm} >{\raggedright\arraybackslash}X p{4cm}}
1 &suppose $\exists x, y, z \in \Z [x \cdot y = z]$ &supposition \\
2 &(suppose x and y are even): $\exists k l \in \Z [2(k) = x \wedge 2(l) = y]$ &supposition + def of even \\
3 &$x \cdot y = z$ &restating 1 \\
4 &$2k \cdot 2l = z$ &algebra \\
5 &$2(k \cdot l) = z$ &algebra \\
6 &$2(k \cdot l)$ is an integer &closure \\
7 &$2(k \cdot l)$ is even &def of even \\
8 &z is even &def of even \\
9 &if two integers are even, then their product is even &conclusions \\
10 &If the product of two integers is odd, then both integers are odd. &contrapositive \\ \hline
11 &(suppose x is even and y is odd): $\exists k l \in \Z [2(k) = x \wedge 2(l) + 1 = y]$ &supposition + def of even + def of odd \\
12 &$x \cdot y = z$ &restating 1 \\
13 &$2k \cdot (2l + 1) = z$ &algebra \\
14 &$ = 4kl + 2k$ &algebra \\
15 &$2kl + k $is an integer &closure \\
16 &$4kl + 2k$ is even &def of even (note this is also expressed as$ 2(2kl + k)$ \\
17 &z is even &def of even \\
18 &if one integer is even and the other is odd, then their product is even &conclusions \\
19 &If the product of two integers is odd, then both integers are odd. &contrapositive \\
\end{tabularx}
\end{table}

\subsubsection{The product of two consecutive integers is even.}

\begin{table}[H]
\centering
\begin{tabularx}{\textwidth}{p{0.8cm} >{\raggedright\arraybackslash}X p{4cm}}
1 &suppose $\exists x, y, z \in \Z [x \cdot y = z \wedge y - x = 1]$ &supposition \\
2 &$2 \mid x \rightarrow 2 \mid y + 1 \vee 2 \mid y \rightarrow 2 \mid x + 1$ &def of evens and odds (even + 1 is odd and vice versa) \\
3 &suppose x is even, making y odd &supposition \\
4 &$\exists k \in \Z [2k = x]$ &def even \\
5 &$\exists l \in \Z [2l +1= y]$ &def odd \\
6 &$x \cdot y = z$ &restating 1 \\
7 &$2k \cdot (2l + 1) = z$ &algebra \\
8 &$4kl + 2k = z$ &algebra \\
9 &$(2 (2kl + k)) = z$ &algebra \\
10 &$2kl + k$ is an integer &closure \\
11 &$(2 (2kl + k)) $ is even &def even \\
12 &z is even &def even \\
13 &The product of two consecutive integers is even. &conclusion \\
14 &suppose y is even, making x odd & \\
15 &$\exists k \in \Z [2k = y]$ &def even \\
16 &$\exists l \in \Z [2l +1= x]$ &def odd \\
17 &$x \cdot y = z$ &restating 1 \\
18 &$2k \cdot (2l + 1) = z$ &algebra \\
19 &The product of two consecutive integers is even. &conclusion (note that the math is exactly as 8 onward, so it is pointless to re-prove) \\
\end{tabularx}
\end{table}

\subsubsection{For any integer $x$, if $6 \mid x^2$, then $4 \mid x^3-x^2$}

\begin{table}[H]
\centering
\begin{tabularx}{\textwidth}{p{0.8cm} >{\raggedright\arraybackslash}X p{4cm}}
1 &suppose $\exists x \in \Z [6 \mid x^2]$ &supposition \\
2 &lemma: $6 \mid x^2 \rightarrow 2 \mid x^2$ &fundamental theorem of arithmetic \\
3 &lemma (cont'd): $2 \mid x^2 \rightarrow 2 \mid x$ &def even + even squares have even square roots \\
4 &lemma: $6 \mid x^2 \rightarrow 3 \mid x^2$ &fundamental theorem of arithmetic \\
5 &lemma (cont'd): $3 \mid x^2 \rightarrow 3 \mid x$ &if a prime divides a square, it divides the root (based on fundamental theorem of arithmetic) \\
6 &$\therefore 6 |x $ &since conclusions drawn in 3 and 5 \\ \hline
7 &let $\exists k \in \Z [6k = x]$ &supposition (existential instantiation) \\
8 &$x^3-x^2 = ((6k)^3 - (6k)^2$ &algebra \\
9 &$= (6^2) (6k^3 - k^2)$ &algebra \\
10 &$= 36 (6k^3 - k^2) = 4 \cdot (9)(6k^3 - k^2) $ &algebra \\
11 &$\therefore 4 | (x^3-x^2) $ &4 is a factor of the latter, proven in 10 \\
12 &For any integer $x$, if $6 \mid x^2$, then $4 \mid x^3-x^2$ &conclusions \\
\end{tabularx}
\end{table}

\subsection{shorthand expansions}
parts of the solutions of this problem set use odd() and even() instead of properly using their definitions. here, I will demonstrate my proficiency in using these despite omitting them from the solution:
\subsubsection{odd()}
given an odd(x) where $x \in \Z$:
$\exists x, k \in \Z [x = 2k +1]$
\subsubsection{even()}
given an even(x) where $x \in \Z$:
$\exists x, k \in \Z [x = 2k]$
\end{document}