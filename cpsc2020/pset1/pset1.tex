\documentclass[12pt, letterpaper]{article}
\title{pset 1}
\author{Jaden Lee\thanks{made in Overleaf}}
\date{Thursday, January 22nd, 2025}
\usepackage{hyperref}
\usepackage{amsmath}
\begin{document}
\maketitle
link to problem: \hyperlink{https://yale.instructure.com/courses/113840/assignments/558946}{https://yale.instructure.com/courses/113840/assignments/558946}
% \neg for neg
% \wedge for and 
% \vee for or
% \rightarrow
% \equiv

\section{Problem 1}
(16 points): Show that 
$(\neg p \vee q) \rightarrow (r \wedge \neg q) $
 and 
 $ (\neg q \wedge (p \vee r )) $
 are logically equivalent
\begin{itemize}
    \item using truth tables (show your intermediate work)
    \item using the list of logical equivalences including the conversion from $\rightarrow$ to $\neg$ and $\vee$ labelling each step with the rule(s) you are using.
\end{itemize}

\subsection{solution}

\[
\begin{aligned}
(\neg p \vee q) \rightarrow (r \wedge \neg q)
&\equiv (\neg p \vee q) \rightarrow (r \wedge \neg q)
&& \text{(given)} \\
&\equiv \neg(\neg p \vee q) \vee (r \wedge \neg q)
&& \text{(substitution**)} \\
&\equiv (p \wedge \neg q) \vee (r \wedge \neg q)
&& \text{(De Morgan)} \\
&\equiv \neg q \wedge (p \vee r)
&& \text{((reverse) distributive )}
\end{aligned}
\]
** since we know that $p \rightarrow q \equiv \neg p \vee q$

\section{problem 2}

(16 points): Let 
\emph{a} be the statement "Alex teaches CS", 
\emph{b} be the statement "Buwan teaches CS", and 
\emph{c} be the statement "Chichima teaches physics."


\begin{itemize}
    \item For each of the following, write the statement form that most closely follows the natural language, and a logically equivalent statement form that does not use \(\wedge\)
        \begin{itemize}
            \item It is not the case that both Alex teaches CS and Chichima teaches physics.
            \item Neither Alex nor Buwan teaches CS.
        \end{itemize}
    \item For each of the following, write the statement form that most closely follows the natural language, and a logically equivalent statement form that uses only $\neg$ $\wedge$ (and parentheses to force order of operations when necessary).
        \begin{itemize}
            \item It is not the case that at least one of Alex or Buwan teaches CS.
            \item If Chichima teaches physics then Alex does not teach CS.
        \end{itemize}
\end{itemize}

\subsection{solution to part 1}
where it is instructed to not use $\wedge$ (and), or $\rightarrow$ (if/then)
\subsubsection{question 1}
"It is not the case that both Alex teaches CS and Chichima teaches physics." translates literally to $$\neg (a \wedge c)$$
however, since we cannot use $\wedge$ (and), we must translate this to an equivalent expression. using distributive law, we find that 
$$\neg (a \wedge c) \equiv \neg a \vee \neg c $$
\subsubsection{question 2}
"Neither Alex nor Buwan teaches CS." translates directly to $$\neg a \vee \neg b$$ However, since this does not use $\wedge$ (and), and since we are using inclusive or (not XOR) this problem is already complete.

\subsection{solution to part 2}
where it is instructed to not use $\vee$ (or), or $\rightarrow$ (if/then)
\subsubsection{question 1}
"It is not the case that at least one of Alex or Buwan teaches CS." translates literally to $$\neg(a \wedge c)$$ note that this does not use $\vee$, and is thus complete.
\subsubsection{question 2}
"If Chichima teaches physics then Alex does not teach CS;" note the presence of the if/then statement. this then can be translated as 
$$ c \rightarrow \neg a $$
from here, we can replace the if/then ("$\rightarrow$")
$$\equiv \neg c \vee \neg a$$
we can see here the $\vee$ (or) has appeared. However, using the distributive law in reverse, we find this expression equivalent to
$$\equiv \neg (c \wedge a)$$
which thus completes the problem



\section{Problem 3}

(4 points): In addition to and, or, not, and if/then, we can define other logical operators. For example, the binary operator NAND ("not and"), with p NAND q (written p $\uparrow$ q) logically equivalent to $\neg (p \wedge q)$. 

the truth table for NAND is then 

\begin{table}[htbp]
\begin{tabular}{|l|l|l|}
\hline
p & q & p $\uparrow$ q \\ \hline
t & t & f              \\ \hline
t & f & t              \\ \hline
f & t & t              \\ \hline
f & f & t              \\ \hline
\end{tabular}
\end{table}

Find a statement form logically equivalent to $\neg (p \wedge \neg p) \vee r$ that uses (possibly more than once) $\uparrow$ as its only logical operator. You will find it helpful to complete the Canvas quiz before attempting this question.

\subsection{solution}
Given that  
$$\neg (p \wedge \neg p) \vee r$$
and that given arbitrary $a$ and $b$:
$$\neg (a \wedge b) \equiv a \uparrow b$$
we can use this definition to say this expression is equivalent to
$$(p \uparrow \neg q) \vee r$$
specifically, using $p$ as $a$ and $\neg q$ as b. 

Furthermore we can simplify further, saying that this is equivalent by negation:
$$\neg (\neg r \wedge \neg (p \uparrow \neg q))$$
which, using the identity mentioned earlier, we can substitute in $\uparrow$s (nand):
$$\neg r \uparrow \neg (p  \uparrow \neg q)$$
to complete this solution, we can make one final identity: that $$\neg a \equiv a \uparrow a$$ 
(this truth table will be expounded later). For now, however, we can use this identity to complete the solution with only $\uparrow$ (nand):
$$(r \uparrow r) \uparrow ((p \uparrow (q \uparrow q)) \uparrow (p \uparrow (q \uparrow q)))$$
as you can see, a section is duplicated to account for the double input of $\uparrow$ (nand)

\newpage
\section{truth tables and figures}
here is a proof of the substitution used earlier:
\subsection{solution 3.1}
$$\neg a \equiv a \uparrow a$$ 


\begin{table}[htbp]
\begin{tabular}{|l|l|l|l|}
\hline
a & $\neg$ a & a $\wedge$ a & a $\uparrow$ a \\ \hline
1 & 0        & 1            & 0             \\ \hline
0 & 1        & 0            & 1             \\ \hline
\end{tabular}
\end{table}
\end{document}